\input{../main.tex}

\begin{document}
	\section*{Работа 2.3.1}	
	\section*{Получение и измерение вакуума}
	\subsection*{Андрей Киркича, Б01-202}
	\n
	\textbf{Цель работы: }
измерить объёмы форвакуумной и высоковакуумной частей установки, определить скорость откачки системы в стационарном режиме, а также по ухудшению и по улучшению вакуума.
	\n\n
	\textbf{В работе используются: }
вакуумная установка с манометрами: масляным, термопарным и ионизационным.
	\n\n
	\subsection*{Теоретичские сведения}
	\begin{itemize}
	\item Классы вакуумных установок:
	\begin{enumerate}
	\item Низковакуумные ($p \sim 10^{-2} - 10^{-3}$ торр)
	\item Высоковакуумные ($p \sim 10^{-4} - 10^{-7}$ торр)
	\item Сверхвысокого вакуума ($p \sim 10^{-8} - 10^{-11}$ торр)
	\end{enumerate}
	\item Низкий вакуум переходит в высокий, когда длина свободного пробега молекул становится сравнима с размерами установки, сверхвысокий вакуум характеризуется крайней важностью процесса адсорбции и десорбции частиц на поверхности вакуумной камеры.
	\end{itemize}
	\subsection*{Экспериментальная установка}
	Установка изготовлена из стекла и состоит форвакуумного баллона (ФБ), высоковакуумного дифузионного насоса (ВН), высоковакуумного баллона (ВБ), маслянного (М) и ионизационного (И) манометров, термопарных манометров ($\w{М}_{1}$ и $\w{М}_{2}$), форвакуумного насоса (ФН) и соединительных кранов. Также в состав установки входят вариатор (автотрансформатор с регулируемым выходным напряжением) или реостат и амперметр для регулирования тока нагревателя диффузионного насоса.
	\img{231-1.png}{2}{Рисунок 1: схема экспериментальной установки}
	\textbf{Краны:}
	\begin{itemize}
	\item $\w{K}_{1}$ - заполнение форвакуумного насоса и вакуумной установки воздухом 
	\item $\w{K}_{2}$ - соединение форвакуумного насоса с установкой или атмосферой
	\item $\w{K}_{3}$ - отделение высоковакуумной части установки от форвакуумной
	\item $\w{K}_{4}$ - соединение между собой колен масляного манометра
	\item $\w{K}_{5}$ и $\w{K}_{6}$ - соединение капилляра с форвакуумной и высоковакуумной частями установки. Суммарный объём обоих кранов $50 \w{ см}^{3}$. Диаметр капилляра $0.9 \w{ мм}$, длина $300 \w{ мм}$.
	\end{itemize}
	\n\n\n\n\n\n\n	
	\textbf{Форвакуумный насос:}
	\img{231-2.png}{2}{Рисунок 2: схема работы форвакуумного насоса}
	В цилиндрической полости размещён ротор так, что он постоянно соприкасается своей верхней частью с корпусом. В диаметральный разрез ротора вставлены две пластины, раздвигаемые пружиной и плотно прижимаемые к поверхности полости.\n\n
	\textbf{Диффузионный насос:}
	\img{231-3.png}{2}{Рисунок 3: схема работы диффузионного насоса}
	Действие основано на диффузии молекул разреженного воздуха в струю паров масла. Наиболее эффективно работает при давлении, когда длина свободного пробега молекул воздуха примерно равна ширине кольцевого зазора между соплом и стенками трубы. В нашей установки диффузионный насос имеет две ступени и соответственно два сопла. Одно сопло вертикальное, другое - горизонтальное.\n\n
	\textbf{Масляный манометр:}\n\n
	Это U-образная трубка, до половины наполненная вязким маслом. Плотность масла $\rho = 0.9 \w{ г/см}^{3}$ мала, поэтому с помощью этого манометра можно измерить лишь небольшие разности давлений.\n\n
	\textbf{Термопарный манометр:}\n\n
	\img{231-4.png}{2}{Рисунок 4: схема термопарного манометра}
	Чувствительным элементом является платинородиевая термопара.\n\n	
	\textbf{Ионизационный манометр:}\n\n
	\img{231-5.png}{2.5}{Рисунок 5: схема ионизационного манометра}	
	Представляет собой трёхэлектродную лампу. Ионный ток в цепи коллектора пропорционален плотности газа и может служить мерой давления.
	\subsection*{Процесс откачки}
	Производительность насоса определяется скоростью откачки $W$ (л/с) - объёмом газа, удаляемого из сосуда при данном давлении за единицу времени.\n\n
	Разделим систему на две части: откачанный объём (в состав которого включим используемые для работы части установки) и насос, к которому, кроме самого насоса, отнесём трубопроводы и краны, через которые производится откачка нашего объёма.
	\begin{itemize}
	\item $Q_{\w{Д}}$ - количество газа, десорбирующегося с поверхности откачиваемого объёма в единицу времени
	\item $Q_{\w{И}}$ - количество газа, проникающего в единицу времени в этот объём извне - через течи
	\item $W$ - скорость откачки насоса, будем считать при этом, что насос сам является источником газа
	\item $Q_{\w{Н}}$ - поток газа, поступающего из насоса назад в откачиваемую систему
	\end{itemize}
	\[-VdP = (PW - Q_{\w{Д}} - Q_{\w{Н}} - Q_{\w{И}})dt\]
	При достижении предельного вакуума ($P_{\w{пр}}$):
	\[\frac{dP}{dt} = 0\]
	\[P_{\w{пр}}W = Q_{\w{Д}} + Q_{\w{Н}} + Q_{\w{И}}\]
	Тогда
	\[Q = \frac{\Sigma Q_i}{P_{\w{пр}}}\]
	В наших условиях все члено можно считать постоянными. Проинтегрируем первое уравнение:
	\[P - P_{\w{пр}} = (P_{0} - P_{\w{пр}}) e^{^{-\frac{W}{V}t}}\]
	\aw{где $P_0$ - начальное давление}
	$P_0$ велико по сравнению с $P_{\w{пр}}$, поэтому можно записать, что:
	\[P - C_{\w{тр}} = P_0 e^{^{-\frac{W}{V}t}}\]
	\n\n
	Закон сложения пропускных способностей при последовательном соединении элементов:
	\[\frac{1}{W} = \frac{1}{W_{\w{Н}}} + \frac{1}{C_1} + \frac{1}{C_2} + ...\]
	\aw{где $W$ - скорость откачки системы, $W_{\w{Н}}$ - скорость откачки насоса, $С_i$ - пропускные способности элементов системы}
	\subsection*{Течение газа через трубу}
	Для газа, протекающего через трубу в кнудсеновском режиме (в условиях высокого вакуума) справедливы формулы:
	\[\frac{d \left (PV \right )}{dt} = \frac{4}{3} r^3 \sqrt{\frac{2 \pi RT}{\mu}} \frac{P_2 - P_1}{L}\]
	\[C_{\w{тр}} = \frac{4}{3}\frac{r^3}{L}\sqrt{\frac{2\pi RT}{\mu}}\]
	\[C_{\w{отв}} = S\frac{\overline{\upsilon}}{4}\]\n\n
	\subsection*{Результаты измерений и обработка данных}
	\subsubsection*{1. Определение объёма форвакуумной и высоковакуумной частей установки}
	$p_0 = 2 \x 10^{-2}$ мм.рт.ст $= 2,67$ Па - начальное давление, полученное в установке\n
	$p_{\w{атм}} = 100.15$ кПа - атмосферное давление\n
	$V_{\w{зап}} = 50 \w{ см}^3$ - объём запертого в установке атмосферного воздуха\n\n
	Ниже приведены высоты столбов масла в масляном манометре при разделении высоковакуумной и форвакуумной частей установки.
	\begin{itemize}
	\item $h_1 = (38.5 \pm 0.1)$ см
	\item $h_2 = (12.0 \pm 0.1)$ см
	\end{itemize}
	Отсюда можно найти разность давлений, полученную в результате высвобождения запертого атмосферного воздуха в форвакуумную часть установки:
	\[\Delta p = \rho g \Delta h = (2301 \pm 18) \w{ Па}\]\n
	Тогда
	\[p_{\w{фв}} = p_0 + \Delta p = (2303 \pm 18) \w{ Па}\]\n
	Затем используем закон Бойля-Мариотта для нахождения объёма форвакуумной части:	
	\[V_{\w{фв}} = \frac{p_{\w{атм}} V_{\w{зап}}}{p_{\w{фв}}} = (2.174 \pm 0.017)\x 10^{-3} \w{ м}^3\]
	\newpage\n
	Аналогично после соединения форвакуумной и высоковакуумной частей с помощью трубки:
	\begin{itemize}
	\item $h_3 = (34.0 \pm 0.1)$ см
	\item $h_4 = (17.0 \pm 0.1)$ см
	\end{itemize}
	\[\Delta p = (1476 \pm 18) \w{ Па}\]
	\[p_{\w{вв}} = (1479 \pm 18) \w{ Па}\]\n
	\[V_{\w{вв}} = \frac{p_{\w{атм}} V_{\w{зап}}}{p_{\w{вв}}} - V_{\w{ фв}} = (1.21 \pm 0.05) \x 10^{-3} \w{ м}^3\]\n
	\subsubsection*{2. Получение высокого вакуума и измерение скорости откачки}
	Установка откачивалась форвакуумным насосом. Были включены термопарные вакууметры.\n\n
	$I = 0.6$ А - ток в лампах\n
	$p = 3 \x 10^{-4}$ мм рт.ст. - давление в высоковакуумном баллоне\n\n
	Затем был закрыт кран $\w{K}_6$, началась высоковакуумная откачка, включён ионизационный манометр, достигнуто предельное давление.\n\n
	\[p_{\w{пр}} = (4.2 \pm 0.1) \x 10^{-5} \w{ мм рт.ст.}\]
	\aw{предельное давление в высоковакуумной части установки после работы ионизационного манометра и дегазации}\n\n
	После измерения предельного давления были проведены опыты по нахождению скорости откачки. Вакуум был ухудшен, а затем восстанавливался. Данные для построения графиков и сами графики зависимости давления от времени представлены ниже.\n\n
\begin{table}[H]
\centering
\begin{tabular}{|cr||cr|}
\hline
\multicolumn{2}{|c||}{№1}       & \multicolumn{2}{|c|}{№2}       \\ \hline
\multicolumn{1}{|c|}{t, c} & p, $10^{-5}$ мм рт.ст. & \multicolumn{1}{c|}{t, с} & p, $10^{-5}$ мм рт.ст. \\ \hline \hline
\multicolumn{1}{|c|}{0}  & 58.0  & \multicolumn{1}{c|}{0}  & 44.0  \\ \hline
\multicolumn{1}{|c|}{1}  & 54.0  & \multicolumn{1}{c|}{1}  & 37.0  \\ \hline
\multicolumn{1}{|c|}{2}  & 44.0  & \multicolumn{1}{c|}{2}  & 30.0  \\ \hline
\multicolumn{1}{|c|}{3}  & 37.0  & \multicolumn{1}{c|}{3}  & 25.0  \\ \hline
\multicolumn{1}{|c|}{4}  & 31.0  & \multicolumn{1}{c|}{4}  & 23.0  \\ \hline
\multicolumn{1}{|c|}{5}  & 28.0  & \multicolumn{1}{c|}{5}  & 20.0  \\ \hline
\multicolumn{1}{|c|}{6}  & 24.0  & \multicolumn{1}{c|}{6}  & 17.0  \\ \hline
\multicolumn{1}{|c|}{7}  & 18.0  & \multicolumn{1}{c|}{7}  & 14.0  \\ \hline
\multicolumn{1}{|c|}{8}  & 16.0  & \multicolumn{1}{c|}{8}  & 12.0  \\ \hline
\multicolumn{1}{|c|}{9}  & 14.0  & \multicolumn{1}{c|}{9}  & 10.0  \\ \hline
\multicolumn{1}{|c|}{10} & 12.0  & \multicolumn{1}{c|}{10} & 9.4 \\ \hline
\multicolumn{1}{|c|}{11} & 11.0  & \multicolumn{1}{c|}{11} & 8.9 \\ \hline
\multicolumn{1}{|c|}{12} & 9.7 & \multicolumn{1}{c|}{12} & 7.9 \\ \hline
\multicolumn{1}{|c|}{13} & 8.8 & \multicolumn{1}{c|}{13} & 7.3 \\ \hline
\multicolumn{1}{|c|}{14} & 8.1 & \multicolumn{1}{c|}{14} & 7.1 \\ \hline
\multicolumn{1}{|c|}{15} & 7.6 & \multicolumn{1}{c|}{15} & 6.7 \\ \hline
\multicolumn{1}{|c|}{16} & 7.1 & \multicolumn{1}{c|}{16} & 6.4 \\ \hline
\multicolumn{1}{|c|}{17} & 6.8 & \multicolumn{1}{c|}{17} & 6.1 \\ \hline
\multicolumn{1}{|c|}{18} & 6.5 & \multicolumn{1}{c|}{18} & 5.9 \\ \hline
\multicolumn{1}{|c|}{19} & 6.2 & \multicolumn{1}{c|}{19} & 5.7 \\ \hline
\multicolumn{1}{|c|}{20} & 6.0  & \multicolumn{1}{c|}{20} & 5.5 \\ \hline
\multicolumn{1}{|c|}{21} & 5.9 & \multicolumn{1}{c|}{21} & 5.5 \\ \hline
\multicolumn{1}{|c|}{22} & 5.7 & \multicolumn{1}{c|}{22} & 5.3 \\ \hline
\end{tabular}
\end{table}
	\aw{Таблица 1: данные, показывающие зависимость давления от времени при улучшении вакуума}
		Масштаб по оси ординат - логарифмический, чтобы можно было найти показатель степени в формуле для давления $-\frac{W}{V}$ как коэффициент наклона $k$ аппроксимирующей прямой. Аппроксимация производилась программно по методу наименьших квадратов только по части значений, так как наблюдалась тенденция к изменению характера зависимости при низких значениях давления.
	\img{plot_2.png}{0.55}{Рисунок 6: график зависимости давления от температуры в логарифмическом масштабе по оси ординат}
	\[W = -k \x V_{\w{вв}}\]
	\n
	$k_1 = (-0.157 \pm 0.004) \w{    } \rightarrow \w{    } W_1 =  (1.90 \pm 0.15) \x 10^{-4} \w{ м}^3/\w{c}$\n
	$k_2 = (-0.150 \pm 0.005) \w{    } \rightarrow \w{    } W_2 = (1.82 \pm 0.14) \x 10^{-4} \w{ м}^3/\w{c}$\n\n
	Таким образом, \[W = (1.86 \pm 0.15) \x 10^{-4} \w{ м}^3/\w{c}\]
	\img{plot_1.png}{0.55}{Рисунок 7: график зависимости давления от температуры в логарифмическом масштабе по оси ординат}
	Затем была найдена величина потока $Q_{\w{н}}$: был перекрыт кран $\w{K}_3$ и при помощи ионизационного вакууметра и секундомера снимались значения далвения с течением времени. Данные для построения графиков и сами графики зависимости давления от времени представлены ниже.\n\n
	Зависимость давления от температуры при ухудшении вакуума - прямая пропорциональная.\n\n
	$\Delta p_1 = (40.0 \pm 0.2) \x 10^{-5}$ мм рт.ст. $\qquad \qquad \Delta p_2 = (39.1 \pm 0.2) \x 10^{-5}$ мм рт.ст.\n
	$\Delta t_1 = 84$ с $\qquad \qquad \qquad \qquad \qquad \qquad \qquad \; \; \Delta t_2 = 84$ c
	\begin{table}[H]
	\centering
\begin{tabular}{|cr||cr|}
\hline
\multicolumn{2}{|c||}{№1}       & \multicolumn{2}{|c|}{№2}       \\ \hline
\multicolumn{1}{|c|}{t, c} & \multicolumn{1}{|c||}{p, $10^{-5}$ мм рт.ст.} & \multicolumn{1}{|c|}{t, c} & \multicolumn{1}{c|}{p, $10^{-5}$ мм рт.ст.} \\ \hline \hline
\multicolumn{1}{|c|}{21}  & 15.0 & \multicolumn{1}{c|}{0}  & 5.9 \\ \hline
\multicolumn{1}{|c|}{24}  & 17.0 & \multicolumn{1}{c|}{3}  & 7.3 \\ \hline
\multicolumn{1}{|c|}{27}  & 18.0 & \multicolumn{1}{c|}{6}  & 8.7 \\ \hline
\multicolumn{1}{|c|}{30}  & 19.0 & \multicolumn{1}{c|}{9}  & 10.0  \\ \hline
\multicolumn{1}{|c|}{33}  & 21.0 & \multicolumn{1}{c|}{12} & 11.0  \\ \hline
\multicolumn{1}{|c|}{36}  & 22.0 & \multicolumn{1}{c|}{15} & 13.0  \\ \hline
\multicolumn{1}{|c|}{39}  & 23.0 & \multicolumn{1}{c|}{18} & 14.0  \\ \hline
\multicolumn{1}{|c|}{42}  & 25.0 & \multicolumn{1}{c|}{21} & 15.0  \\ \hline
\multicolumn{1}{|c|}{45}  & 26.0 & \multicolumn{1}{c|}{24} & 16.0  \\ \hline
\multicolumn{1}{|c|}{48}  & 28.0 & \multicolumn{1}{c|}{27} & 18.0  \\ \hline
\multicolumn{1}{|c|}{51}  & 29.0 & \multicolumn{1}{c|}{30} & 19.0  \\ \hline
\multicolumn{1}{|c|}{54}  & 30.0 & \multicolumn{1}{c|}{33} & 21.0  \\ \hline
\multicolumn{1}{|c|}{57}  & 32.0 & \multicolumn{1}{c|}{36} & 22.0  \\ \hline
\multicolumn{1}{|c|}{60}  & 33.0 & \multicolumn{1}{c|}{39} & 24.0  \\ \hline
\multicolumn{1}{|c|}{63}  & 35.0 & \multicolumn{1}{c|}{42} & 25.0  \\ \hline
\multicolumn{1}{|c|}{66}  & 36.0 & \multicolumn{1}{c|}{45} & 26.0  \\ \hline
\multicolumn{1}{|c|}{69}  & 37.0 & \multicolumn{1}{c|}{48} & 28.0  \\ \hline
\multicolumn{1}{|c|}{72}  & 39.0 & \multicolumn{1}{c|}{51} & 29.0  \\ \hline
\multicolumn{1}{|c|}{75}  & 40.0 & \multicolumn{1}{c|}{53} & 30.0  \\ \hline
\multicolumn{1}{|c|}{78}  & 41.0 & \multicolumn{1}{c|}{57} & 32.0  \\ \hline
\multicolumn{1}{|c|}{81}  & 43.0 & \multicolumn{1}{c|}{60} & 33.0  \\ \hline
\multicolumn{1}{|c|}{84}  & 44.0 & \multicolumn{1}{c|}{63} & 35.0  \\ \hline
\multicolumn{1}{|c|}{87}  & 46.0 & \multicolumn{1}{c|}{66} & 36.0  \\ \hline
\multicolumn{1}{|c|}{90}  & 47.0 & \multicolumn{1}{c|}{69} & 37.0  \\ \hline
\multicolumn{1}{|c|}{93}  & 49.0 & \multicolumn{1}{c|}{72} & 39.0  \\ \hline
\multicolumn{1}{|c|}{96}  & 50.0 & \multicolumn{1}{c|}{75} & 40.0  \\ \hline
\multicolumn{1}{|c|}{99}  & 52.0 & \multicolumn{1}{c|}{78} & 42.0  \\ \hline
\multicolumn{1}{|c|}{102} & 53.0 & \multicolumn{1}{c|}{81} & 43.0  \\ \hline
\multicolumn{1}{|c|}{105} & 55.0 & \multicolumn{1}{c|}{84} & 45.0  \\ \hline
\end{tabular}
\end{table}
\aw{Таблица 2: данные, показывающие зависимость давления от времени при ухудшении вакуума}
	\[V_{\w{вв}} \Delta p = (Q_{\w{Д}} + Q_{\w{И}})\Delta t\]
	\[Q_{\w{Д}} + Q_{\w{И}} = \frac{V_{\w{вв}} \Delta p}{\Delta t}\]\n
	\img{plot_3.png}{0.55}{Рисунок 8: график зависимости давления от температуры}
	\[Q_{\w{Н}} = p_{\w{пр}}W - (Q_{\w{Д}} + Q_{\w{И}}) = p_{\w{пр}}W - \frac{V_{\w{вв}} \Delta p}{\Delta t}\]\n
	$Q_{\w{Н}_1} = (2.7 \pm 1.7) \x 10^{-7} \w{ Па} \x \w{м}^3/\w{c}$\n
	$Q_{\w{Н}_2} = (2.9 \pm 1.7) \x 10^{-7} \w{ Па} \x \w{м}^3/\w{c}$\n\n
	Тогда
	\[Q_{\w{Н}} = (2.8 \pm 1.7) \x 10^{-7} \w{ Па} \x \w{м}^3/\w{c}\]
	\img{plot_4.png}{0.55}{Рисунок 9: график зависимости давления от температуры}
	Для расчёта пропускной способности трубки используются следующие величины:\n\n
	$L = 10.8 \w{ см}$ - длина трубки\n
	$r = 0.4 \w{ cм}$ - радиус трубки\n
	$T = (298 \pm 1) \w{ K}$ - температура в комнате\n
	\[C_{\w{тр}} = (5.80 \pm 0.03) \x 10^{-4} \w{ м}^3/\w{c}\]\n
	Скорость откачки составляет 32\% от полученной пропускной способности трубки.\n\n
	Последней задачей работы стало определение производительности насоса. Была создана искусственная течь (открыт кран $\w{K}_6$), и вакуум ухудшился.\n\n
	$p_{\w{уст}} = 1.1 \x 10^{-4} \w{ мм рт.ст.}$ - установившееся давление\n
	$p_{\w{фв}} = 3.0 \x 10^{-4} \w{ мм рт.ст.}$ - давление со стороны форвакуумной части\n\n
	\[p_{\w{пр}} W = Q_1 \w{,} \qquad p_{\w{уст}} W = Q_1 + \frac{d \left (pV \right )}{dt}\]
	Учитывая приведённое выше соотношение, получим:
	\[W = \frac{4}{3} \frac{r^3}{L} \sqrt{\frac{2 \pi RT}{\mu}} \frac{p_{\w{фв}} - p_{\w{уст}}}{p_{\w{уст}} - p_{\w{пр}}} = (1.6 \pm 0.3) \x 10^{-3} \w{ м}^3/\w{c}\]\n
	
	\subsubsection*{3. Расчёт погрешностей}
	Следующие величины считаем константами без погрешности:
	\begin{itemize}
	\item $p_0 = 2 \x 10^{-2}$ мм рт.ст.
	\item $p_{\w{атм}} = 100.15$ кПа
	\item $V_{\w{зап}} = 50 \w{ см}^3$
	\item $g = 9.81 \w{ м}/\w{с}^2$
	\item $\pi = 3.14$
	\item $R = 8.31 \w{ Дж}/\w{К} \x \w{моль}$
	\item $\mu = 29 \w{ г}/\w{моль}$
	\item $r = 0.4$ см
	\item $L = 10.8$ см
	\item $\rho = 885 \w{ кг}/\w{м}^3$
	\end{itemize}\n
	$\sigma(T) = 1$ К\n
	$\sigma(h) = 0.1$ см\n
	$\sigma(\Delta h) = 0.2$ см\n
	$\sigma(\Delta p) = \rho g \x \sigma(\Delta h) = 18$ Па\n
	$\sigma(p_{\w{фв}}) = \sigma(p_{\w{вв}}) = \sigma(\Delta p) = 18$ Па\n
	$\sigma(V_{\w{фв}}) = \frac{p_{\w{атм}} V_{\w{зап}}}{p_{\w{фв}}^2} \x \sigma(p_{\w{фв}}) = 0.017 \x 10^{-3} \w{ м}^3$\n
	$\sigma(V_{\w{вв}}) = \sigma(V_{\w{фв}}) + \frac{p_{\w{атм}} V_{\w{зап}}}{p_{\w{вв}}^2} \x \sigma(p_{\w{вв}}) = 0.05 \x 10^{-3}$\n
	$\sigma(p_{\w{пр}}) = \sigma(p_{\w{уст}}) = \sigma(\w{прибора}) = 0.1 \x 10^{-5}$ мм рт.ст.\n
	$\sigma(k)$ - по методу наименьших квадратов считается программно\n
	$\sigma(W) = \sigma(k) \x V_{\w{вв}} - k \x \sigma(V_{\w{вв}})$\n
	$\sigma(\Delta p (\w{при ухудшении вакуума})) = 0.2 \x 10^{-5}$ мм рт.ст.\n
	$\Delta t = 2$ с\n
	$\Delta Q_{\w{Н}} = \sigma(p_{\w{пр}}) \x W + \sigma(W) \x p_{\w{пр}} + \frac{\Delta p}{\Delta t} \x \sigma(V_{\w{вв}}) + \frac{V_{\w{вв}}}{\Delta t} \x \sigma(\Delta p) + \frac{V_{\w{вв}} \x \Delta p}{\Delta t^2} \x \sigma(\Delta t)$\n
	$C_{\w{тр}} = \frac{4}{3} \x \frac{r^3}{L} \x \sqrt{\frac{2\pi R}{\mu}} \x \frac{\sigma(T)}{2\sqrt{T}}$\n
	$\sigma(W (\w{второй способ})) = \frac{4}{3} \x \frac{r^3}{L} \x \sqrt{\frac{2\pi R}{\mu}} \x \frac{p_{\w{фв}} - p_{\w{уст}}}{p_{\w{уст}} - p_{\w{пр}}} \x \frac{\sigma(T)}{2\sqrt{T}} + \frac{4}{3} \x \frac{r^3}{L} \x \sqrt{\frac{2\pi RT}{\mu}} \x \frac{1}{p_{\w{уст}} - p_{\w{пр}}} \x 2\sigma(p_{\w{пр}}) + \frac{4}{3} \x \frac{r^3}{L} \x \sqrt{\frac{2\pi RT}{\mu}} \x \frac{p_{\w{фв}} - p_{\w{уст}}}{(p_{\w{уст}} - p_{\w{пр}})^2} \x 2\sigma(p_{\w{пр}})$\n
	Погрешность среднего арифметического считается как среднее арифметическое погрешностей
	\n
	\subsection*{Вывод}
	Объём форвакуумной части установки - $(2.174 \pm 0.017)\x 10^{-3} \w{ м}^3$\n
	Объём высоковакуумной части установки - $(1.21 \pm 0.05)\x 10^{-3} \w{ м}^3$\n
	Предельное давление, полученное в установке - $(4.2 \pm 0.1) \x 10^{-5}$ мм рт.ст\n
	Скорость откачки системы - $(1.86 \pm 0.15) \x 10^{-4} \w{ м}^3/\w{c}$. Это значение не совпадает со скоростью откачки, полученной теоретически ($(1.6 \pm 0.2) \x 10^{-3} \w{ м}^3/\w{c}$). Это может быть связано с изменениями температуры внутри установки, вызванными работой приборов, а также плохой герметичностью установки. Возможна также некорректная работа манометров.
 	\end{document}
