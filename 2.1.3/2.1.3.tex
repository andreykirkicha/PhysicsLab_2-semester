\documentclass[12pt,a4paper]{article}
\usepackage[utf8]{inputenc}
\usepackage[T2A]{fontenc}
\usepackage[english, russian]{babel}
\usepackage{amsmath}
\usepackage{amsfonts}
\usepackage{amssymb}
\usepackage{titleps}
\usepackage{geometry}
\usepackage{hyperref}
\usepackage{float}
\usepackage{graphicx}
\usepackage{multirow}
\usepackage{hhline}

\newcommand{\w}[1]{\text{#1}}
\newcommand{\und}[1]{\underline{#1}}
\newcommand{\img}[3]{
	\begin{figure}[H]
	\begin{center}
	\includegraphics[scale=#2]{#1}
	\end{center}
	\begin{center}
 	\textit{#3}
	\end{center}
	\end{figure}
}
\newcommand{\aw}[1]{
	\begin{center}
	\textit{#1}
	\end{center}
	\n
}
\newcommand{\be}[1]{
	\begin{center}
	\boxed{#1}
	\end{center}
}
\newcommand{\beb}[1]{
	\begin{equation}
	\boxed{#1}
	\end{equation}
}
\newcommand{\n}{\hfill \break}
\newcommand{\x}{\cdot}

\begin{document}
	\section*{Работа 2.1.3}	
	\section*{Определение показателя адиабаты по скорости звука в газе}
	\subsection*{Андрей Киркича, Б01-202, МФТИ, 2023}
	\n
	\textbf{Цель работы: }
измерение частоты колебаний и длины волны при резонансе звуковых колебаний в газе, заполняющем трубу; определение показателя адиабаты с помощью уравнения состояния идеального газа
	\n\n
	\textbf{В работе используются: }
звуковой генератор (ГЗ); электронный осциллограф (ЭО); микрофон; телефон; теплоизолированная труба, обогреваемая водой из термостата; баллон со сжатым углекислым газом; газгольдер
	\n\n
	\subsection*{Теоретичские сведения}\n
	\beb{\gamma = \frac{\mu}{RT} c^2}
	\beb{L = n \frac{\lambda}{2} \w{ , } n \in \mathbb{N}}
	\beb{c = \lambda f}
	\beb{f_{k} = \frac{c}{\lambda_{k}} = \frac{c}{2L} (n + k) = f_1 + \frac{c}{2L} k \w{ , } k \w{ - номер резонанса}}
	\subsection*{Методика измерений}\n
	В теплоизолированной трубе с помощью телефона Т возбуждаются звуковые колебания, которые улавливаются микрофоном М. В качестве источника переменной ЭДС, приводящей в движение мембрану телефона, используется звуковой генератор. Сигнал с микрофона выводится на электронный осциллограф. Воздух в трубе нагревается водой из термостата.
	\img{pic_1.png}{2}{Рисунок 1: схема экспериментальной установки}
	В трубе при определённой температуре с помощью генератора подбирается резонансная частота, при этом на осциллографе улавливается повышение амплитуды сигнала в несколько раз. Из графика зависимости разности $k+1$-ой и $1$-ой резонансных частот от номера резонанса $k$ можно найти скорость звука при данной температуре. Зная скорость звука, находим показатель адиабаты.
	\subsection*{Результаты измерений}\n
	Было проведено четыре измерения при разных значениях температуры. Результаты измерений приведены ниже.

\begin{table}[H]
\centering
	\resizebox{340pt}{!}{
\begin{tabular}{|rr||rr||rr||rr|}
\hline
\multicolumn{2}{|c||}{$t_0 = 24 ^\circ C$} & \multicolumn{2}{c||}{$t_1 = 40 ^\circ C$} & \multicolumn{2}{c||}{$t_2 = 60 ^\circ C$} & \multicolumn{2}{c|}{$t_3 = 80 ^\circ C$} \\ \hline
\multicolumn{1}{|c|}{$k$} &
  \multicolumn{1}{c||}{$f_{k + 1} - f_{1}$, Гц} &
  \multicolumn{1}{c|}{$k$} &
  \multicolumn{1}{c||}{$f_{k + 1} - f_{1}$, Гц} &
  \multicolumn{1}{c|}{$k$} &
  \multicolumn{1}{c||}{$f_{k + 1} - f_{1}$, Гц} &
  \multicolumn{1}{c|}{$k$} &
  \multicolumn{1}{c|}{$f_{k + 1} - f_{1}$, Гц} \\ \hline \hline
\multicolumn{1}{|r|}{1}   & 230   & \multicolumn{1}{r|}{1}   & 233   & \multicolumn{1}{r|}{1}   & 238   & \multicolumn{1}{r|}{1}   & 248   \\ \hline
\multicolumn{1}{|r|}{2}   & 460   & \multicolumn{1}{r|}{2}   & 466   & \multicolumn{1}{r|}{2}   & 480   & \multicolumn{1}{r|}{2}   & 487   \\ \hline
\multicolumn{1}{|r|}{3}   & 689   & \multicolumn{1}{r|}{3}   & 704   & \multicolumn{1}{r|}{3}   & 720   & \multicolumn{1}{r|}{3}   & 742   \\ \hline
\multicolumn{1}{|r|}{4}   & 918   & \multicolumn{1}{r|}{4}   & 938   & \multicolumn{1}{r|}{4}   & 966   & \multicolumn{1}{r|}{4}   & 991   \\ \hline
\multicolumn{1}{|r|}{5}   & 1148  & \multicolumn{1}{r|}{5}   & 1174  & \multicolumn{1}{r|}{5}   & 1208  & \multicolumn{1}{r|}{5}   & 1244  \\ \hline
\multicolumn{1}{|r|}{6}   & 1378  & \multicolumn{1}{r|}{6}   & 1413  & \multicolumn{1}{r|}{6}   & 1455  & \multicolumn{1}{r|}{6}   & 1496  \\ \hline
\multicolumn{1}{|r|}{7}   & 1612  & \multicolumn{1}{r|}{7}   & 1649  & \multicolumn{1}{r|}{7}   & 1699  & \multicolumn{1}{r|}{7}   & 1748  \\ \hline
\multicolumn{1}{|r|}{8}   & 1843  & \multicolumn{1}{r|}{8}   & 1888  & \multicolumn{1}{r|}{8}   & 1947  & \multicolumn{1}{r|}{8}   & 1997  \\ \hline
\end{tabular}
}
\end{table}
\aw{Таблица 1: разность частот при каждом резонансе}
	\img{plot_1.png}{0.55}{Рисунок 2: график зависимости разности частот при температуре $t_0 = 24 ^\circ C$}
	\img{plot_2.png}{0.55}{Рисунок 3: график зависимости разности частот при температуре $t_1 = 40 ^\circ C$}
	\img{plot_3.png}{0.55}{Рисунок 4: график зависимости разности частот при температуре $t_2 = 60 ^\circ C$}
	\img{plot_4.png}{0.55}{Рисунок 5: график зависимости разности частот при температуре $t_3 = 80 ^\circ C$}
	Графики отражают прямую пропорциональную зависимость между разностью $k+1$-ой и $1$-ой резонансных частот и номером резонанса $k$. Эта зависимость была предсказана формулой (5).\n\n
	Линейная аппроксимация производилась программно по методу наименьших квадратов.\n\n
	Коэффициент наклона графика равен $r = \frac{c(T)}{2L}$, для каждой температуры коэффициенты приведены в таблице ниже.
	\newpage
\begin{table}[H]
\centering
\begin{tabular}{|r|r|r|r|}
\hline
\multicolumn{1}{|c|}{$t, ^\circ C$} & \multicolumn{1}{c|}{$r, 1/$c} & \multicolumn{1}{c|}{$c$, м/с} & \multicolumn{1}{c|}{$\gamma$} \\ \hline
24 & 230,3 & 340,8 & $1,365 \pm 0,009 $ \\ \hline
40 & 236,5 & 350,0 & $1,366 \pm 0,008$ \\ \hline
60 & 244,1 & 361,3 & $1,368 \pm 0,010$ \\ \hline
80 & 250,8 & 371,2 & $1,362 \pm 0,011$ \\ \hline
\end{tabular}
\end{table}
\aw{Таблица 2: коэффициенты наклона графиков, скорости звука и показатели адиабаты для каждого значения температуры}
	В данной работе длина трубкой была постоянной:
	\be{L = (740 \pm 1) \w{ мм}}
	Видно, что с увеличением температуры растёт и скорость звука. Формула (1) показывает, что скорость звука прямо пропорциональна квадратному корню из температуры.\n\n
	Среднее значения показателя адиабаты:
	\be{\gamma = 1,37 \pm 0,01}

	\subsection*{Расчёт погрешностей}\n
	\begin{itemize}
	\item $\mu, R$ считаем константами без погрешности
	\item $\sigma_{t} = 1 ^\circ C$
	\item $\sigma_{L} = 1 \w{ мм}$
	\item Погрешность углового коэффициента $\sigma_{r}$ считается программно из метода наименьших квадратов
	\item $\sigma_{c} = 2L \x \sigma_{r} + 2r \x \sigma_{L}$
	\item $\sigma_{\gamma} = \frac{\mu}{R} \x (\frac{2c}{T} \x \sigma_{c} + \frac{c^2}{T^2} \x \sigma_{T})$
	\end{itemize}
	
	\subsection*{Вывод}\n
	Показатель адиабаты оказался равен $1,37 \pm 0,01$. Это значение не сходится с табличной величиной - $1.4$ (разность составляет 2\% от табличного значения). Расхождение может быть связано со слишком грубой настройкой резонансной частоты, обусловленной конструктивными особенностями осциллографа, и, как следствие, неточным определением скорости звука.\n\n
	При малых изменениях температуры показатель адиабаты остаётся постоянным.\n\n
	Резонансная частота зависит от номера резонанса линейно, характер зависимости сохраняется при разных температурах.\n\n
	Скорость звука прямо пропорциональна квадратному корню из температуры.
\end{document}
