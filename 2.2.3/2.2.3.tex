\documentclass[12pt,a4paper]{article}
\usepackage[utf8]{inputenc}
\usepackage[T2A]{fontenc}
\usepackage[english, russian]{babel}
\usepackage{amsmath}
\usepackage{amsfonts}
\usepackage{amssymb}
\usepackage{titleps}
\usepackage{geometry}
\usepackage{hyperref}
\usepackage{float}
\usepackage{graphicx}
\usepackage{multirow}
\usepackage{hhline}

\newcommand{\w}[1]{\text{#1}}
\newcommand{\und}[1]{\underline{#1}}
\newcommand{\img}[3]{
	\begin{figure}[H]
	\begin{center}
	\includegraphics[scale=#2]{#1}
	\end{center}
	\begin{center}
 	\textit{#3}
	\end{center}
	\end{figure}
}
\newcommand{\aw}[1]{
	\begin{center}
	\textit{#1}
	\end{center}
	\n
}
\newcommand{\be}[1]{
	\begin{center}
	\boxed{#1}
	\end{center}
}
\newcommand{\beb}[1]{
	\begin{equation}
	\boxed{#1}
	\end{equation}
}
\newcommand{\n}{\hfill \break}
\newcommand{\x}{\cdot}

\begin{document}
	\section*{Работа 2.2.3}	
	\section*{Измерение теплопроводности воздуха при атмосферном давлении}
	\subsection*{Андрей Киркича, Б01-202, МФТИ, 2023}
	\n
	\textbf{Цель работы: }
измерить коэффициент теплопроводности воздуха при атмосферном давлении в зависимости от температуры
	\n\n
	\textbf{В работе используются: }
цилиндрическая колба с натянутой по оси нитью; термостат; вольтметр и амперметр (цифровые мультиметры); эталонное сопротивление; источник постоянного напряжения; реостат или магазин сопротивлений
	\n\n
	\subsection*{Теоретичские сведения}
	\textit{Теплопроводность} - процесс передачи тепловой энергии от нагретых частей системы к холодным за счёт хаотического движения частиц среды.\n\n
	\beb{\vec{q} = -\kappa \x \vec{\nabla} T}
	\aw{закон Фурье}
	\beb{Q = \frac{2\pi L}{\ln r_0/r_1} \kappa \Delta T}\n
	\beb{R(t) = R_{273} \x (1 + \alpha t)}\n
	\subsection*{Методика измерений}
	\img{pic_1.png}{2}{Рисунок 1: схема экспериментальной установки}
	\img{pic_2.jpg}{0.13}{Рисунок 2: электрическая схема для измерения сопротивления}
	На оси полой цилиндрической размещена металлическая нить. Полость трубки заполнена воздухом, стенки помещены в кожух, через который пропускается вода из термостата. Для предотвращения конвекции труба расположена вертикально.\n\n
	Металлическая нить служит как источником тепла, так и датчиком температуры.\n\n
	По пропускаемому через нить постоянному току и напряжению вычисляется мощность нагрева по закону Джоуля-Ленца и сопротивление по закону Ома.\n\n
	Сопротивление нити является однозначной функцией её температуры. Эта зависимость измеряется по экстраполяции мощности нагрева к нулю. Если материал нити известен, то зависимость можно найти по справочным данным.\n\n
	Чем выше ток, тем с большей точностью он будет измерен. Однако при этом квадратично возрастает выделяющаяся на резисторе мощность, а значит, температура резистора становится выше, чем объекта, температуру которого надо измерить.\n\n
	В данной работе для решения этой проблемы строится нагрузочная кривая в диапазоне температур $20\w{-}80 ^\circ C$. В этом интервале кривую можно приблизить линейной функцией (3). Экстраполяция этой зависимости для $Q \rightarrow 0$ позволит определить температуру нити по значению её сопротивления $R$ при произвольной мощности нагрева.\n\n
	В работе использовалась установка со следующими характеристиками:
	\begin{itemize}
	\item Материал нити - платина
	\item $L = 0.4$ м - длина нити
	\item $2r_0 = 0.7$ см - внутренний диаметр цилиндрической трубки
	\item $2r_1 = 0.05$ мм - диаметр платиновой нити
	\end{itemize}
	\newpage
	\subsection*{Результаты измерений}
	Было проведено пять измерений нагрузочных кривых. Результаты измерений и графики зависимости $R_{\w{н}}(Q)$ при разных значениях температуры представлены ниже.
	\begin{table}[H]
	\centering
	\resizebox{450pt}{!}{
\begin{tabular}{|r|r|r|r|r|r|r||r|r|r|r|r|r|r|}
\hline
\multicolumn{1}{|c|}{$T, ^\circ$C} &
  \multicolumn{1}{c|}{$\alpha$} &
  \multicolumn{1}{c|}{$R_{\w{м}}$, Ом} &
  \multicolumn{1}{c|}{$U$, В} &
  \multicolumn{1}{c|}{$I$, мА} &
  \multicolumn{1}{c|}{$Q$, Вт} &
  \multicolumn{1}{c||}{$R_{\w{н}}$, Ом} &
  \multicolumn{1}{c|}{$T, ^\circ$C} &
  \multicolumn{1}{c|}{$\alpha$} &
  \multicolumn{1}{c|}{$R_{\w{м}}$, Ом} &
  \multicolumn{1}{c|}{$U$, В} &
  \multicolumn{1}{c|}{$I$, мА} &
  \multicolumn{1}{c|}{$Q$, Вт} &
  \multicolumn{1}{c|}{$R_{\w{н}}$, Ом} \\ \hline \hline
22 & 0.01 & 180.0  & 0.38 & 18.75  & 0.0071 & 20.3 &    & 0.40  & 11.6 & 2.18 & 93.79  & 0.2045 & 23.2 \\ \cline{2-7} \cline{9-14} 
   & 0.05 & 69.5 & 0.86 & 40.25  & 0.0346 & 21,4 &    & 0.45 & 9.8  & 2.28 & 97.83  & 0.2231 & 23.3 \\ \cline{2-7} \cline{9-14} 
   & 0.10  & 43.3 & 1.13 & 55.20   & 0.0624 & 20.5 &    & 0.50  & 8.3  & 2.37 & 101.45 & 0.2404 & 23.4 \\ \cline{2-7} \cline{9-14} 
   & 0.20  & 24.7 & 1.50  & 74.65  & 0.1120 & 20.1 &    & 0.60  & 5.8  & 2.54 & 108.05 & 0.2744 & 23.5 \\ \cline{2-7} \cline{9-14} 
   & 0.30  & 16.5 & 1.85 & 88.12  & 0.1630 & 21.0  &    & 0.70  & 3.9  & 2.69 & 113.60  & 0.3056 & 23.7 \\ \cline{2-14} 
   & 0.40  & 11.6 & 2.09 & 98.61  & 0.2061 & 21.2 & 67 & 0.01 & 180.0  & 0.43 & 18.46  & 0.0079 & 23.3 \\ \cline{2-7} \cline{9-14} 
   & 0.45 & 9.8  & 2.19 & 103.04 & 0.2257 & 21.3 &    & 0.05 & 69.5 & 0.91 & 38.96  & 0.0355 & 23.4 \\ \cline{2-7} \cline{9-14}
   & 0.50  & 8.3  & 2.29 & 107.04 & 0.2451 & 21.4 &    & 0.10  & 43.3 & 1.24 & 52.79  & 0.0655  & 23.5  \\ \cline{2-7} \cline{9-14} 
   & 0.60  & 5.8  & 2.47 & 114.32 & 0.2824  & 21.6 &    & 0.20  & 24.7 & 1.68 & 70.40   & 0.1183 & 23.9 \\ \cline{1-7} \cline{9-14} 
37 & 0.01 & 180.0  & 0.39 & 18.65  & 0.0073 & 20.9 &    & 0.30  & 16.5 & 1.98 & 82.36  & 0.1631 & 24.0  \\ \cline{2-7} \cline{9-14} 
   & 0.05 & 69.5 & 0.85 & 39.80   & 0.0338  & 21.4 &    & 0.40  & 11.6 & 2.22 & 91.52  & 0.2032 & 24.3 \\ \cline{2-7} \cline{9-14} 
   & 0.10  & 43.3 & 1.17 & 54.36  & 0.0636 & 21.5 &    & 0.45 & 9.8  & 2.32 & 95.39  & 0.2213 & 24.3 \\ \cline{2-7} \cline{9-14} 
   & 0.20  & 24.7 & 1.59 & 73.18  & 0.1164 & 21.7 &    & 0.50  & 8.3  & 2.41 & 98.85  & 0.2382 & 24.3 \\ \cline{2-7} \cline{9-14} 
   & 0.30  & 16.5 & 1.89 & 86.13  & 0.1628 & 21.9 &    & 0.60  & 5.8  & 2.58 & 105.13 & 0.2712 & 24.5 \\ \cline{2-7} \cline{9-14} 
   & 0.40  & 11.6 & 2.09 & 98.61  & 0.2061 & 21.2 &    & 0.70  & 3.9  & 2.73 & 110.43 & 0.3015 & 24.7 \\ \cline{2-14} 
   & 0.45 & 9.8  & 2.24 & 100.36 & 0.2248 & 22.3 & 80 & 0.01 & 180.0  & 0.45 & 18.38  & 0.0083 & 24.4 \\ \cline{2-7} \cline{9-14} 
   & 0.50  & 8.3  & 2.33 & 104.15 & 0.2427  & 22.4 &    & 0.05 & 69.5 & 0.94 & 38.60   & 0.0363 & 24.3 \\ \cline{2-7} \cline{9-14} 
   & 0.60  & 5.8  & 2.50  & 111.05 & 0.2776 & 22.5 &    & 0.10  & 43.3 & 1.28 & 52.14  & 0.0667 & 24.5 \\ \cline{2-7} \cline{9-14} 
   & 0.70  & 3.9  & 2.66 & 116.90  & 0.3110 & 22.8 &    & 0.20  & 24.7 & 1.71 & 69.27  & 0.1185 & 24.6 \\ \cline{2-7} \cline{9-14} 
   & 0.80  & 2.4  & 2.79 & 141.30  & 0.3942 & 19.7 &    & 0.30  & 16.5 & 2.02 & 80.81  & 0.1632 & 24.9 \\ \cline{1-7} \cline{9-14} 
52 & 0.01 & 180.0  & 0.42 & 18.56  & 0.0078 & 22.6 &    & 0.40  & 11.6 & 2.25 & 89.66  & 0.2017 & 25.0 \\ \cline{2-7} \cline{9-14} 
   & 0.05 & 69.5 & 0.88 & 39.38  & 0.0347 & 22.3 &    & 0.45 & 9.8  & 2.35 & 93.36  & 0.2194 & 25.1 \\ \cline{2-7} \cline{9-14} 
   & 0.10  & 43.3 & 1.21 & 53.57  & 0.0648  & 22.6 &    & 0.50  & 8.3  & 2.44 & 96.69  & 0.2359 & 25.2 \\ \cline{2-7} \cline{9-14} 
   & 0.20  & 24.7 & 1.63 & 71.77  & 0.1170 & 22.7 &    & 0.60  & 5.8  & 2.61 & 105.13 & 0.2744 & 24.8 \\ \cline{2-7} \cline{9-14} 
   & 0.30  & 16.5 & 1.94 & 84.21  & 0.1634 & 23.0 &    & 0.70  & 3.9  & 2.75 & 107.79 & 0.2964 & 25.5 \\ \hline
\end{tabular}
}
\end{table}
	\aw{Таблица 1: реузльтаты измерений}
	В некоторых графиках наблюдаются сильные отклонения нескольких точек - они не вписываются в общий характер зависимости. Поэтому в аппроксимации эти точки не учитывались.
	\img{plot_1.png}{0.4}{Рисунок 3: график зависимости $R_{\w{н}}(Q)$ при температуре $T_1 = 22 ^o C$}
	\img{plot_2.png}{0.4}{Рисунок 4: график зависимости $R_{\w{н}}(Q)$ при температуре $T_2 = 37 ^o C$}
	\img{plot_3.png}{0.4}{Рисунок 5: график зависимости $R_{\w{н}}(Q)$ при температуре $T_3 = 52 ^o C$}\n
	Видно, что графики представляют из себя отражение линейной зваисимости между $R_{\w{н}}$ и $Q$.
	\img{plot_4.png}{0.4}{Рисунок 6: график зависимости $R_{\w{н}}(Q)$ при температуре $T_3 = 67 ^o C$}
	\img{plot_5.png}{0.4}{Рисунок 7: график зависимости $R_{\w{н}}(Q)$ при температуре $T_3 = 80 ^o C$}
	\n
	По приведённым выше данным можно найти значения сопротивлений при $Q = 0$ и построить их зависимость от $T$.
	\begin{table}[H]
	\centering
\begin{tabular}{|r|r|}
\hline
\multicolumn{1}{|c|}{$T$, К} & \multicolumn{1}{c|}{$R$, Ом} \\ \hline \hline
295                     & 20.20                   \\ \hline
310                     & 21.07                  \\ \hline
325                     & 22.35                  \\ \hline
340                     & 23.22                  \\ \hline
353                     & 24.36                  \\ \hline
\end{tabular}
\end{table}
\aw{Таблица 2: зависимость сопротивления нити от её температуры}
	\img{plot_6.png}{0.5}{Рисунок 8: график зависимости сопротивления нити от температуры}\n
	Экстраполируя график к $T = 273$ К получаем $R_{273} = (19 \pm 1)$ Ом. Теперь можно найти температурный коэффициент сопротивления материала нити:\n
	\be{\alpha = \frac{1}{R_{273}} \frac{dR}{dT} = (3.9 \pm 0.4) \x 10^{-3} \w{ K}^{-1}}\n\n
	Затем находим отношения $\frac{dQ}{d(\Delta T)} = \frac{dR / dT}{dR / dQ}$ и, используя формулу (2), коэффициенты теплопроводности.
	\begin{table}[H]
	\centering
\begin{tabular}{|r|r|r|r|}
\hline
\multicolumn{1}{|c|}{$T$, К} & \multicolumn{1}{c|}{$\frac{dR}{dQ}$, $\frac{\w{Ом}}{\w{Вт}}$} & \multicolumn{1}{c|}{$\frac{dQ}{d(\Delta T)}$, $\frac{\w{Вт}}{\w{К}}$} & \multicolumn{1}{c|}{$\kappa$, $\frac{\w{Вт}}{\w{м} \x \w{К}}$} \\ \hline \hline
295 & 4.85 & 0.0148  & 0.029 \\ \hline
310 & 5.41  & 0.0132 & 0.026  \\ \hline
325 & 4.20 & 0.0171 & 0.034 \\ \hline
340 & 4.97 & 0.0144 & 0.028  \\ \hline
353 & 3.95 & 0.0181 & 0.036 \\ \hline
\end{tabular}
\end{table}
\aw{Таблица 3: значения $\frac{dR}{dQ}$, $\frac{dQ}{d(\Delta T)}$ и $\kappa$ для каждого значения температуры}
	Ниже представлен график зависимости $\kappa (T)$.
	\img{plot_7.png}{0.3}{Рисунок 9: график зависимости теплопроводности воздуха от температуры}\n
	Сложно говорить о какой-либо зависимости.
	
	\subsection*{Расчёт погрешностей}
	\begin{itemize}
	\item $r_0, r_1, L$ считаем константами без погрешности
	\item $\sigma_{T} = 0.1$ К
	\item $\sigma_{R_{\w{м}}} = 0.1$ Ом
	\item $\sigma_{U} = 0.01$ В
	\item $\sigma_{I} = 0.01$ мА
	\item $\sigma_{Q} = U \x \sigma_{I} + I \x \sigma_{U} \approx 0.01Q$
	\item $\sigma_{R_{\w{н}}} = \frac{1}{I} \x \sigma_{U} + \frac{U}{I^2} \x \sigma_{I} \approx 0.01 R_{\w{н}}$
	\item $\sigma_{\frac{dR}{dQ}}$, $\sigma_{R}$, $\sigma_{R_{273}}$ и $\sigma_{\frac{dR}{dT}}$ определяются программно из метода наименьших квадратов
	\item $\sigma_{\alpha} = \frac{1}{R_{273}} \x \sigma_{\frac{dR}{dT}} + \frac{dR/dT}{R_{273}^2} \x \sigma_{R_{273}}$
	\item $\sigma_{\frac{dQ}{d(\Delta T)}} = \frac{1}{dR/dQ} \x \sigma_{\frac{dR}{dT}} + \frac{dR/dT}{(dR/dQ)^2} \x \sigma_{\frac{dR}{dQ}} \approx 0.04 \frac{dQ}{d(\Delta T)}$
	\item $\sigma_{\kappa} = \frac{\ln r_0 / r_1}{2\pi L} \x \sigma_{\frac{dQ}{d(\Delta T)}} \approx 0.04 \kappa$
	\end{itemize}
	
	\subsection*{Вывод}
	Результат работы противоречивый. С одной стороны, из нагрузочных кривых получили хорошее приближение графика зависимости $R(T)$ к прямой линии, что и ожидалось увидеть (формула (3)), а также совпадение температурного коэффициента сопротивления материала нити в пределах погрешности с табличным значением ($\alpha = (3.9 \pm 0.4) \x 10^{-3}$ К $^{-1}$, табличное - $3.729 \x 10^{-3}$ К $^{-1}$). С другой стороны, зависимость полученного из этих данных коэффициента теплопроводности от температуры нельзя каким-либо образом охарактеризовать. Проблема кроется в угловых коэффициентах прямых, описывающих зависимость $R_{\w{н}} (Q)$. На графиках есть резко выпадающие из общей зависимости точки. После изменения параметров система переходит в стационарное состояние не сразу, а по истечении нескольких десятков секунд. При измерениях значения тока испытывали большие по амплитуде флуктуации, а показания снимали, не дожидаясь установления равновесия. Также мог повлиять набор сопротивлений - некоторые сопротивления могли отличаться в реальности от тех, которые планировалось устанавливать. Прибор не самый новый: проводящие элементы могли износиться или их химический состав изменился. Стоило дожидаться установления равновесия и провести больше измерений для каждой температуры, чтобы зависимость приняла более уверенный характер.
\end{document}
