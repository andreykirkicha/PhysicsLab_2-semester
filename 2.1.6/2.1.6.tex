\documentclass[12pt,a4paper]{article}
\usepackage[utf8]{inputenc}
\usepackage[T2A]{fontenc}
\usepackage[english, russian]{babel}
\usepackage{amsmath}
\usepackage{amsfonts}
\usepackage{amssymb}
\usepackage{titleps}
\usepackage{geometry}
\usepackage{hyperref}
\usepackage{float}
\usepackage{graphicx}
\usepackage{multirow}
\usepackage{hhline}

\newcommand{\w}[1]{\text{#1}}
\newcommand{\und}[1]{\underline{#1}}
\newcommand{\img}[3]{
	\begin{figure}[H]
	\begin{center}
	\includegraphics[scale=#2]{#1}
	\end{center}
	\begin{center}
 	\textit{#3}
	\end{center}
	\end{figure}
}
\newcommand{\aw}[1]{
	\begin{center}
	\textit{#1}
	\end{center}
	\n
}
\newcommand{\be}[1]{
	\begin{center}
	\boxed{#1}
	\end{center}
}
\newcommand{\beb}[1]{
	\begin{equation}
	\boxed{#1}
	\end{equation}
}
\newcommand{\n}{\hfill \break}
\newcommand{\x}{\cdot}

\begin{document}
	\section*{Работа 2.1.6}	
	\section*{Эффект Джоуля-Томсона}
	\subsection*{Андрей Киркича, Б01-202, МФТИ, 2023}
	\n
	\textbf{Цель работы: }
	определение изменения температуры углекислого газа при протекании через малопроницаемую перегородку при разных начальных значениях давления и температуры; вычисление по результатам опытов коэффициентов Ван-дер-Ваальса
	\n\n
	\textbf{В работе используются: }
	трубка с пористой перегородкой; труба Дьюара; термостат; термометры; дифференциальная термопара; микровольтметр; балластный баллон; манометр
	\n\n
	\subsection*{Теоретичские сведения}\n
	\und{Эффект Джоуля-Томсона} - изменение температуры газа. медленно протекающего из области высокого в область низкого давления в условиях хорошей тепловой изоляции.
	\beb{A_{1} - A_{2} = \left ( U_{2} + \frac{\mu \upsilon_{2}^{2}}{2}\right ) - \left ( U_{1} + \frac{\mu \upsilon_{1}^{2}}{2} \right )}
	\beb{H_{1} - H_{2} = \frac{1}{2} \mu (\upsilon_{2}^{2} - \upsilon_{1}^{2})}
	\beb{\Delta T = \frac{\mu}{2 C_{p}}(\upsilon_{2}^{2} - \upsilon_{1}^{2})}
	\beb{\mu_{\w{Дж-Т}} = \frac{\Delta T}{\Delta P} \approx \frac{\frac{a}{RT} - b}{C_{p}}}
	\beb{T_{\w{инв}} = \frac{2a}{Rb}}
	\beb{T_{\w{кр}} = \frac{8a}{27Rb}}
	\subsection*{Методика измерений}\n
	В работе исследуется изменение температуры углекислого газа при медленном его течении по трубе с пористой перегородкой.
	\img{pic_1.png}{1.5}{Рисунок 1: схема экспериментальной установки}
	\n
	Углекислый газ под повышенным давлением поступает в трубку через змеевик 5 из балластного баллона 6. Медный змеевик омывается водой и нагревает протекающий через него газ. Температура воды измеряется термометром $\w{T}_{\w{в}}$, размещённым в термостате. Требуемая температура воды устанавливается и поддерживается при помощи контактного термометра $\w{Т}_{\w{к}}$. Давление газа в трубке измеряется манометром М и регулируется вентилем В. Манометр измеряет разность между давлением внутри трубки и наружным (атмосферным), то есть непосредственно разность $\Delta P = P_{1} - P_{2}$. Разность температур газа измеряется термопарой медь-константан. Для уменьшения теплоотвода трубка с пористой перегородкой помещена в трубу Дьюара 3, стенки которой посеребрены для уменьшения теплоотдачи, связанной с излучением. Для уменьшения теплоотдачи за счёт конвекции один конец трубы Дьюара уплотнён кольцом 4, а другой закрыт пробкой 10 из пенопласта.
	\subsection*{Результаты измерений}\n
	Было проделано три серии измерений разности температур и разности давлений газа на концах трубки при различных температурах воды в термостате. Результаты представлены ниже.
	\img{plot_1.png}{0.5}{Рисунок 2: график зависимости разности температуры от разности давлений при температуре 20 $^o$C}
	\img{plot_2.png}{0.37}{Рисунок 3: график зависимости разности температуры от разности давлений при температуре 30 $^o$C}
	\img{plot_3.png}{0.37}{Рисунок 4: график зависимости разности температуры от разности давлений при температуре 50 $^o$C}
	\begin{table}[H]
	\centering
	\resizebox{450pt}{!}{
\begin{tabular}{|r|r|r|r|r|r|r|r|r|}
\hline
\multicolumn{1}{|c|}{$s$, $\frac{\w{мкВ}}{\w{К}}$} &
  \multicolumn{1}{c|}{$T$, $^o$C} &
  \multicolumn{1}{c|}{$\Delta P$, $\frac{\w{кгс}}{\w{см}^2}$} &
  \multicolumn{1}{c|}{$\Delta P$, атм} &
  \multicolumn{1}{c|}{$U_0$, мВ} &
  \multicolumn{1}{c|}{$U$, мВ} &
  \multicolumn{1}{c|}{$\Delta U$, мкВ} &
  \multicolumn{1}{c|}{$\Delta T$, К} &
  \multicolumn{1}{c|}{$\mu_{\w{Дж-Т}}$, $\frac{\w{К}}{\w{атм}}$} \\ \hline \hline
39.8 & 20 & -4.1 & -4.0   & 0.011 & -0.159 & -170 & -4.27 & $1.17 \pm 0.09$ \\ \cline{3-4} \cline{6-8}
     &    & -3.7 & -3.6 &  & -0.141 & -152 & -3.82  &         \\ \cline{3-4} \cline{6-8}
     &    & -3.3 & -3.2 &  & -0.121 & -132 & -3.32 &         \\ \cline{3-4} \cline{6-8}
     &    & -2.9 & -2.8 &  & -0.096 & -107 & -2.69 &         \\ \cline{3-4} \cline{6-8}
     &    & -2.4  & -2.4 &  & -0.088 & -99  & -2.49 &         \\ \hline
40.7 & 30 & -4.1 & -4.0   & 0.018 & -0.143 & -161 & -3.96 & $1.02 \pm 0.02$ \\ \cline{3-4} \cline{6-8}
     &    & -3.7 & -3.6 &  & -0.127 & -145 & -3.56 &         \\ \cline{3-4} \cline{6-8}
     &    & -3.3 & -3.2 &  & -0.109 & -127 & -3.12 &         \\ \cline{3-4} \cline{6-8}
     &    & -2.9 & -2.8 &  & -0.092 & -110 & -2.70  &         \\ \cline{3-4} \cline{6-8}
     &    & -2.4  & -2.4 &  & -0.076 & -94  & -2.31 &         \\ \hline
42.5 & 50 & -4.1 & -4.0   & 0.025 & -0.095 & -120 & -2.82 & $0.77 \pm 0.03$ \\ \cline{3-4} \cline{6-8}
     &    & -3.7 & -3.6 &  & -0.081 & -106 & -2.49 &         \\ \cline{3-4} \cline{6-8}
     &    & -3.3 & -3.2 &  & -0.066 & -91  & -2.14 &         \\ \cline{3-4} \cline{6-8}
     &    & -2.9 & -2.8 &  & -0.054 & -79  & -1.86 &         \\ \cline{3-4} \cline{6-8}
     &    & -2.4  & -2.4 &  & -0.043 & -68  & -1.60     &         \\ \hline
\end{tabular}
}
\end{table}
	\aw{Таблица 1: результаты измерений}
	\n
	Затем по полученным данным можно рассчитать константы $a$ и $b$ в уравнении Ван-дер-Ваальса. используя формулу (4) и взяв две разные температуры для решения полученной системы.
	\be{
	\begin{cases}
	a = \frac{(\mu_1 - \mu_2) C_{P} R T_1 T_2}{2(T_2 - T_1)}\\
	b = \frac{C_{P} (\mu_2 T_2 - \mu_1 T_1)}{T_1 - T_2}
	\end{cases}
	}
	В работе эти величины были посчитаны для двух пар температур - 20-30 $^o$С и 30-50 $^o$С. $C_{P} = 4R$.
\begin{table}[H]
\centering
\begin{tabular}{|r|r|r|}
\hline
\multicolumn{1}{|c|}{Пара температур, $^o$C} & \multicolumn{1}{c|}{$a$, $\frac{\w{Н} \x \w{м}^4}{\w{моль}^2}$} & \multicolumn{1}{c|}{b} \\ \hline
20-30                  & $1.7 \pm 0.3$              & $990 \pm 190$ \\ \hline
30-50                  & $1.8 \pm 0.3$               & $1090 \pm 190$           \\ \hline
\end{tabular}
\end{table}
	\n
	По формуле (5) получаем температуру инверсии:
	\be{T_{\w{инв}} = (400 \pm 100) \w{ К}}
	\subsection*{Расчёт погрешностей}
	\begin{itemize}
	\item $R, C_{P}, s$ считаем константами без погрешности
	\item $\sigma_{T} = 0.3$ $^o$C
	\item $\sigma_{\Delta P} = 0.1$ $\frac{\w{кгс}}{\w{см}^2}$
	\item $\sigma_{U} = 0.001$ мВ
	\item $\sigma_{\Delta U} = \sigma_{U} \sqrt{2} \approx 1.4$ мкВ
	\item $\sigma_{\Delta T} = \Delta T \x \frac{\sigma_{\Delta U}}{\Delta U} \approx 1 \%$ от $\Delta T$
	\item $\sigma_{\mu_{\w{Дж-Т}}}$ рассчитывается программно методом наименьших квадратов
	\item $\sigma_{a} = a \sqrt{\frac{\sigma_{\mu_{1}}^2 + \sigma_{\mu_{2}}^2}{(\mu_{1} - \mu_{2})^2} + \left ( \frac{\sigma_{T_{1}}}{T_{1}} \right )^2 + \left ( \frac{\sigma_{T_{2}}}{T_{2}} \right )^2 + \frac{\sigma_{T_{1}}^2 + \sigma_{T_{2}}^2}{(T_{1} - T_{2})^2}}$
	\item $\sigma_{b} = b \sqrt{\frac{\sigma_{T_{1}}^2 + \sigma_{T_{2}}^2}{(T_{1} - T_{2})^2} + \frac{(\mu_{1} T_{1})^2 ( \varepsilon_{\mu_{1}}^2 + \varepsilon_{T_{1}}^2) + (\mu_{2} T_{2})^2 ( \varepsilon_{\mu_{2}}^2 + \varepsilon_{T_{2}}^2)}{(\mu_{2} T_{2} - \mu_{1} T_{1})^2}}$
	\item $\sigma_{T_{\w{инв}}} = T_{\w{инв}} \x \sqrt{\left ( \frac{\sigma_{a}}{a} \right )^2 + \left ( \frac{\sigma_{b}}{b} \right )^2}$
	\end{itemize}
	\subsection*{Вывод}
	Коэффициенты уравнения Ван-дер-Ваальса для углекислого газа и температура инверсии в пределах погрешности не сходятся. Это связано с тем, что разность давлений в эксперименте была слишком большой, эффект не был дифференциальным. Также могли повлиять потери энергии в трубке, но несущественно. Стоит использовать более точные методы для измерения температуры, чтобы снимать показания с малой разностью давлений. Коэффициент Джоуля-Томсона в данной работе имел место быть, так как снятые зависимости разности давлений от разности температур были линейными. В процессе газ охлаждался. 
\end{document}
