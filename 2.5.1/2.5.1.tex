\input{../main.tex}

\begin{document}
	\section*{Работа 2.5.1}	
	\section*{Измерение коэффициента поверхностного натяжения жидкости}
	\subsection*{Андрей Киркича, Б01-202, МФТИ, 2023}
	\n
	\textbf{Цель работы: }
	измерение температурной зависимости  коэффициента поверхностного натяжения дистиллированной воды с использованием известного коэффициента поверхностного натяжения спирта, определение полной поверхностной энергии  и теплоты, необходимой для изотермического образования единицы  поверхности жидкости  при различной температуре
	\n\n
	\textbf{В работе используются: }
	прибор Ребиндера с термостатом и микроманометром; исследуемые жидкости; стаканы
	\n\n
	\subsection*{Теоретичские сведения}\n
	Наличие поверхностного слоя приводит к различию давлений по разные стороны от искривленной границы раздела двух сред.  Для сферического пузырька с воздухом  внутри жидкости избыточное давление даётся формулой Лапласа:\n
	\beb{\Delta p = p_{\w{внутр}} - p_{\w{внеш}} = \frac{2\sigma}{r}}
	\aw{где $\sigma$ - коэффициент поверхностного натяжения, $r$ - радиус кривизны поверхности раздела двух фаз}
	\subsection*{Методика измерений}\n
	В работе измеряется даление $\Delta p$, необходимое для выталкивания в жидкость пузырька воздуха.
	\img{pic_1.png}{2}{Рисунок 1: схема экспериментальной установки}\n
	Исследуемая жидкость (дистиллированная вода) наливается в сосуд В. Тестовая жидкость (этиловый спирт) наливается  в сосуд Е.  При измерениях  колбы герметично закрываются  пробками.   Через одну из двух пробок  проходит полая металлическая игла С. Этой пробкой закрывается сосуд, в котором  проводятся измерения. Верхний конец иглы открыт в атмосферу, а нижний погружен в жидкость. Другой сосуд герметично закрывается второй пробкой. При создании достаточного  разряжения воздуха в колбе с иглой пузырьки воздуха начинают пробулькивать через жидкость. Поверхностное натяжение можно определить по величине разряжения $\Delta p$, необходимого для прохождения пузырьков, при известном радиусе иглы.\n\n
	Разряжение в системе создается с помощью аспиратора А. Кран $\w{К}_{2}$ разделяет две полости аспиратора. Верхняя полость при закрытом кране $\w{К}_{2}$  заполняется водой. Затем кран $\w{К}_{2}$ открывают и заполняют водой  нижнюю полость  аспиратора.  Разряжение воздуха создается в нижней полости  при открывании крана $\w{К}_{1}$, когда  вода вытекает из неё по каплям. В колбах В и С, соединённых трубками с нижней полостью аспиратора,  создается такое же пониженное давление. Разность давлений в полостях с разряженным воздухом и атмосферой измеряется спиртовым микроманометром. \n\n
Для стабилизации температуры исследуемой жидкости через рубашку D колбы В непрерывно прогоняется вода из термостата.\n\n
	\subsection*{Результаты измерений}\n
	Измерения проводились для разных значений температуры в диапазоне $20-80$ $^o$C. По следующей формуле определялось давление внутри:
	\beb{p = C \x N \x \frac{\rho_{\w{залитого}}}{\rho_{\w{указанная}}} \x K \x 9.81}
	\aw{где $N$ - количество делений по шкале манометра, $C$ - поправочный множитель, $K$ - постоянная угла наклона, $\rho_{\w{залитого}}$ - плотность спирта, залитого в прибор, $\rho_{\w{указанная}}$ - плотность спирта, указанная на приборе}
	Для данной установки:
	\begin{itemize}
	\item $C = 1$
	\item $\rho_{\w{залитого}} = 0.8049 \frac{\w{г}}{\w{см}^3}$
	\item $\rho_{\w{указанная}} = 0.8095 \frac{\w{г}}{\w{см}^3}$
	\item $K = 0.2$
	\item $r = 0.6$ мм
	\end{itemize}
	\n	
	Внешнее давление считается по следующей формуле:
	\beb{p_{\w{внеш}} = \rho g \Delta h}
	$\Delta h = 16.5$ мм - глубина погружения иглы под поверхность\n
	В расчётах учитывается зависимость плотности воды от температуры\n\n
	Затем можно найти избыточное давление $\Delta p = p - p_{\w{внеш}}$ и по формуле (1) рассчитать коэффициент поверхностного натяжения. Ниже приведены результаты измерений. По этим значениям можно построить графики зависимости коэффициента поверхностного натяжения $\sigma (T)$, теплоты образования единицы поверхности жидкости $q (T)$, поверхностной энергии единицы площади $u_{\w{пов}} (T)$ от температуры.
	\begin{table}[H]
	\centering
\begin{tabular}{|r|r|r|r|r|r|}
\hline
\multicolumn{1}{|c|}{$T$, $^o$C} & \multicolumn{1}{c|}{$N$, делений} & \multicolumn{1}{c|}{$\Delta p$, Па} & \multicolumn{1}{c|}{$\sigma$, $10^{-2} \frac{\w{Н}}{\w{м}}$} & \multicolumn{1}{c|}{$q$, $\frac{\w{мДж}}{\w{м}^2}$} & \multicolumn{1}{c|}{$u_{\w{пов}}$, $\frac{\w{мДж}}{\w{м}^2}$} \\ \hline \hline
20.0 & 197 & 220 $\pm$ 10 & 6.7 $\pm$ 0.4 & 2.4 $\pm$ 0.4 & 69 $\pm$ 4\\ \hline
26.0 & 195 & 219 $\pm$ 10 & 6.6 $\pm$ 0.4 & 3.2 $\pm$ 0.6 & 69 $\pm$ 4\\ \hline
30.3 & 195 & 219 $\pm$ 10 & 6.6 $\pm$ 0.4 & 3.7 $\pm$ 0.7 & 69 $\pm$ 4\\ \hline
35.3 & 195 & 219 $\pm$ 10 & 6.6 $\pm$ 0.4 & 4.3 $\pm$ 0.8 & 70 $\pm$ 4\\ \hline
40.3 & 193 & 216 $\pm$ 10 & 6.5 $\pm$ 0.4 & 4.9 $\pm$ 0.9 & 70 $\pm$ 4\\ \hline
46.0 & 192 & 214 $\pm$ 10 & 6.4 $\pm$ 0.4 & 5.6 $\pm$ 0.9 & 70 $\pm$ 4\\ \hline
50.1 & 192 & 215 $\pm$ 10 & 6.4 $\pm$ 0.4 & 6.1 $\pm$ 1.1 & 71 $\pm$ 4\\ \hline
55.0 & 189 & 209 $\pm$ 10 & 6.3 $\pm$ 0.4 & 6.7 $\pm$ 1.1 & 69 $\pm$ 4\\ \hline
60.0 & 186 & 204 $\pm$ 10 & 6.1 $\pm$ 0.3 & 7.3 $\pm$ 1.2 & 68 $\pm$ 4\\ \hline
\end{tabular}
\end{table}
\aw{Таблица 1: результаты измерений}
	Ниже представлены упомянутые графики.
	\img{plot_1.png}{0.4}{Рисунок 2: график зависимости коэффициента поверхностного натяжения от температуры}\n
	Отсюда получаем температурный коэффициент:
	\be{k = \frac{d \sigma}{dT} = -(1.2 \pm 0.2) \x 10^{-4} \frac{\w{Н}}{\w{м} \x \w{К}}}
	\n
	Построим зависимости от температуры:
	\begin{itemize}
	\item Теплоты образования единицы поверхности жидкости $q = -T \x \frac{d\sigma}{dT}$
	\item Поверхностной энергии единицы площади $u_{\w{пов}} = \sigma - T \x \frac{d\sigma}{dT}$
	\end{itemize}
	\img{plot_2.png}{0.5}{Рисунок 3: график зависимости теплоты образования единицы поверхности жидкости от температуры}
	\img{plot_3.png}{0.5}{Рисунок 4: график зависимости поверхностной энергии единицы площади от температуры}	
	\subsection*{Расчёт погрешностей}
	\begin{itemize}
	\item $\rho_{\w{залитого}}$, $\rho_{\w{указанная}}$, $K$, $C$, $g$, $\rho$ считаем константами без погрешности
	\item $\sigma_{N} = 1$ деление
	\item $\sigma_{p} = C\frac{\rho_{\w{залитого}}}{\rho_{\w{указанная}}} K \x 9.81 \x \sigma_{N}$
	\item $\sigma_{\Delta h} = 1$ мм
	\item $\sigma_{p_{\w{внеш}}} = \rho \x g \x \sigma_{\Delta h}$
	\item $\sigma_{\Delta p} = \sqrt{\sigma_{p}^2 + \sigma_{p_{\w{внеш}}}^2}$
	\item $\sigma_{r} = 0.05$ мм
	\item $\sigma_{\sigma} = \frac{1}{2} \sigma \sqrt{\left (\frac{\sigma_{\Delta p}}{\Delta p} \right )^2 + \left (\frac{\sigma_{r}}{r} \right )^2}$
	\item $\sigma_{k}$ вычисляется программно методом наименьших квадратов
	\item $\sigma_{T} = 0.1$ К
	\item $\sigma_{q} = q \x \sqrt{\left (\frac{\sigma_{T}}{T} \right )^2 + \left (\frac{\sigma_{k}}{k} \right )^2}$
	\item $\sigma_{u_{\w{пов}}} = \sqrt{\sigma_{\sigma}^2 + \sigma_{q}^2}$
	\end{itemize}
	\subsection*{Вывод}
	Коэффициент поверхностного натяжения с ростом температуры линейно убывает. Поверхностная энергия единицы площади поверхности жидкости не зависит от температуры. Об этом говорит совпадение значений энергии на выбранном интервале температур в пределах погрешности и практически горизонтальная приближающая прямая на графике зависимости $u_{\w{пов}}(T)$. Среднее значение $u_{\w{ср}} = (69 \pm 4) \frac{\w{мДж}}{\w{м}^2}$. Теплота, необходимая для изотермического образования единицы поверхности жидкости, прямо пропорциональна температуре.
\end{document}
